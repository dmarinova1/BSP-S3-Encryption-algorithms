\section{Introduction ($\pm$ 5\% total words)}
% no \IEEEPARstart
We are living in the information age enabled by computing and communication technologies whose rapid evolution is almost taken for granted today. This is an age dense with electronic connectivity, eavesdropping and fraud. We send messages around the world instantaneously and we do not want them to be intercepted or stolen. Thus, information is an asset that has precious value like any other. Society’s reliance on a changing panoply of information technologies and technology-enabled services, the increasing global nature of commerce and business, and the ongoing desire to protect freedoms all suggest that future needs for data security is vital.  Network – based attacks spring up every day, distance does not matter anymore since data has been digitalized and there is the issue of privacy being collected, processed and misrepresented. For this reason, the disciplines of cryptography and network security have matured. Thus, internet security is complex yet fascinating. To develop a security mechanism (algorithm or protocol) one must always consider potential attacks, take countermeasures, act like an intruder and go against one’s own intuition and exploit eventual weaknesses in the algorithm. 

We dive into the terminology that governs our topic in the next paragraph. The word “cryptography” comes from two Greek words meaning “secret writing” [1]. Cryptography is a vast, rich subject. It is the art and science of concealing meaning, of keeping the enciphered information secret and of designing and analyzing encryption schemes. It involves mathematics as a supporting tool. Cryptanalysis is the breaking of codes and uses statistical and mathematical approaches. The areas of cryptography and cryptanalysis together are called cryptology. The basic component of cryptography is a cryptosystem which is based on substitution (it changes characters in the plaintext to produce the ciphertext) and permutation (an ordered sequence of elements in a finite set with each character appearing only once).
If someone wants to break the ciphertext, knowing the algorithm but not the specific cryptographic key,  there are 3 main types of attacks [Ref: Stallings, 7th ed, p.90]: 
-	ciphertext only attack – ciphertext known, the goal is to find the corresponding plaintext. The key may be found as well if possible. (A Caesar cipher is susceptible to a statistical ciphertext-only attack)
-	known plaintext attack – both ciphertext and plaintext and known; the goal is to find the key
-	chosen plaintext attack – specific plaintexts are enciphered in order to find out the corresponding ciphertexts and the goal is to find the key.
	a good cryptosystem protects against all three types of attacks. 

We often stumble upon impending security threats such as information leakage; integrity violation; denial of service; illegitimate usage; trojan horses; insider attacks; problems related to access or authentication control. 
Thus, there are instilled security requirements of utmost importance. We present them as follows: confidentiality; integrity; authentication; availability; access control; non-repudiation, which are often intermingled. Hence, one of the mechanisms promising to ensure security is encryption, along with digital signatures and hashing. 

In order to make information secret, we use a cipher – an algorithm that converts plain text into ciphertext, which is in unreadable form, unless we have a key that lets us convert back the cipher. The process of making text secret is called encryption, and the reverse process is called decryption.

There are two types of encryption: symmetric and asymmetric. For this project, we shall focus mainly on the symmetric, also called “conventional”. It comprises a single key for both encryption and decryption and it is the preferred choice when we need to encrypt large amounts of data. On the other hand, asymmetric cryptography, also called “public key” and “two-key”, consists of two keys: public and private. The public key is used to encrypt the message, whereas the private key is used to reverse the encryption. This method cannot deal with large quantities of data due to the runtime of supporting processes employed. However, it is more secure when we want to transfer over an insecure channel. Nevertheless, the choice to be made on which system to use for encryption and decryption depends on the purposes and on each algorithms’ functions. 
