\section{Introduction ($\pm$ 5\% total words)}
% no \IEEEPARstart
We are living in the information age enabled by computing and communication technologies, an age dense with electronic connectivity, eavesdropping and fraud. For all that, when we send messages around the world instantaneously, we do not want them to be intercepted or stolen. Thus, information is an asset that has precious value like any other. Future needs for data security is vital, given society's reliance on a changing panoply of information technologies and technology-enabled services, the increasing global nature of commerce and business, and the ongoing desire to protect freedoms. We all use the Internet, all the data is becoming congested, therefore, how do we maintain privacy, integrity, online-identity? Network – based attacks spring up every day, remoteness is not a problem anymore since data has been digitalized, not to mention the issue of privacy being collected, processed and misrepresented, all conducting the vulnerability of Internet security. For this reason, the disciplines of Cryptography and Network Security have matured. They protect financial flows and medical records, defend state secrets and the corporate sort, and make the e-commerce industry possible. Without cryptography, credit-card details, bank transfers, emails and the like would zip around the Internet unprotected, for anyone to see or steal. From obscure science, Cryptography today is one of the major pillars of modern cyber reality. To develop a security mechanism (algorithm or protocol) one must always consider potential attacks, take countermeasures, act like an intruder, bet against one’s own intuition and exploit eventual weaknesses in the algorithm. 

In order to understand the principles of cryptography, we dive into the terminology that governs our topic in the next paragraph. The word \emph{cryptography} comes from two Greek words meaning \emph{secret writing} \cite{wikicrypto}. Cryptography is the art and science of concealing meaning, of keeping the enciphered information secret and designing and analyzing encryption schemes. Mathematics is the foundation on which modern encryption rests. Cryptanalysis is the breaking of encoded data and uses statistical and mathematical approaches. The areas of cryptography and cryptanalysis together are called cryptology. The basic component of cryptography is a cryptosystem which is based on substitution and permutation. The former changes characters in the plaintext to produce the ciphertext; the latter is an ordered sequence of elements in a finite set with each character appearing only once. In order to make information secret, we use a cipher – a cryptographic algorithm that converts plaintext (data in readable form) into ciphertext (a message in unreadable protected form), unless we have a key that lets us convert back the cipher. The process of making text secret is called encryption, and the reverse process is called decryption.

There are two types of encryption: symmetric and asymmetric. For this project, we shall focus mainly on the symmetric, also called \emph{conventional}. It comprises a single key for both encryption and decryption and it is the preferred choice when we need to encrypt large amounts of data because it is faster and less power hungry mainly on the decryption side. Some of the encryption algorithms that use symmetric key encryption are: DES (Data Encryption Standard),Triple DES, AES (Advanced Encryption Standard), Blowfish, Twofish. One of the disadvantages is related to transferring secure files because the same key has to be used for encryption and decryption. The sender must find a secure way to provide the recipient with the key so that the latter is able to decrypt the files. Otherwise, they risk placing the key in wrong hands, leading to decryption of the encrypted files. Not to mention the high impossibility to distribute a key when the file transfer environment involves multiple users dispersed around the world. 

On the other hand, asymmetric cryptography, also called \emph{public key} and \emph{two-key}, makes use of two keys: public and private. The public key is used to encrypt the message, whereas the private key is used to reverse the encryption. Most widely employed asymmetric key algorithms are RSA and DSA. However, this method is computationally costly \cite{wikikey}, thus, cannot deal with large quantities of data due to: the runtime of supporting processes employed; the more storage space required and the more time needed to move over a link. However, it is more secure when we want to transfer over an insecure channel. Moreover, it does not have the issue with multiple user key distribution, as long as the private key is kept secret.

It is not easy to compare the cryptographic strengths of both key encryptions since having longer key lengths in asymmetric encryption does not infer impossibility to break. Nevertheless, the choice to be made on which system to use for encryption and decryption depends on the purposes, requirements, ease of distribution and on the functions of each algorithm. Modern file transfer systems employ a hybrid versions of both symmetric and asymmetric key cryptography as they have their own advantages. For example, we classify SSL and SSH as hybrid cryptosystems. 

If someone wants to break the ciphertext, knowing the algorithm but not the specific cryptographic key,  we present 3 main types of attacks as there are others as well. And a good cryptosystem protects against all three types of attacks \cite{stallings2017} 
\begin{itemize}
\item Ciphertext only attack : ciphertext known, the goal is to find the corresponding plaintext. The key may be found as well if possible. (A Caesar cipher is susceptible to a statistical ciphertext-only attack)
\end{itemize}
\begin{itemize}
\item Known plaintext attack : both ciphertext and plaintext and known; the goal is to find the key
\end{itemize}
\begin{itemize}
\item Chosen plaintext attack : specific plaintexts are enciphered in order to find out the corresponding ciphertexts and the goal is to find the key.
\end{itemize}


We often stumble upon impending security threats such as information leakage, integrity violation, denial of service, illegitimate usage, trojan horses, insider attacks, problems related to access or authentication control. 
Thus, there are instilled security requirements of utmost importance to mention just a few: confidentiality, integrity, authentication, availability, access control, non-repudiation, often intermingled to provide further security. Hence, one of the mechanisms promising to ensure security is encryption, along with digital signatures and hashing. 



