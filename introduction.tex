\section{Introduction ($\pm$ 5\% total words)}
% no \IEEEPARstart
Communication technologies have turned the past decades into an information age, dense with electronic connectivity, eavesdropping and fraud. We witness an influx of sensitive information breaches. The reason being is the lack of strong access control measures. Various methods exist for hacking and circumventing sensitive data, e.g social engineering attacks, distributed brute forcing. It is worth mentioning, the use of the same password to access different accounts that grant easy account access to determined scammers. We should not fall short in addressing today’s threat landscape. For all that, when we send private data around the world instantaneously, we do not want it to be intercepted, redirected or stolen. Thus, information is an asset that has precious value like any other. Modern society relies on an evolving panoply of information technologies and technology-enabled services. Future needs of data security is of utmost importance given the increasing global nature of commerce, business, communication, eHealth and eGovernment. We all use the Internet, data is becoming congested. Therefore, how do we maintain privacy, integrity, online-identity? Network – based attacks spring up every day and distance for hackers is not a problem anymore since data has been digitalized. There are various factors contributing to the vulnerability of Internet security. For this reason, the disciplines of Cryptography and Network Security have matured. Thanks to them financial flows and medical records are being protected. Cryptography and Network Security combined defend state secrets and the corporate sort, and make the e-commerce industry possible. Without cryptography, credit-card details, bank transfers would zip around the Internet unprotected, for anyone to see or steal. From obscure science, Cryptography has become one of the major pillars of modern cyber reality. To develop a security mechanism (algorithm or protocol) one must always consider potential attacks, take countermeasures, role play as an intruder and exploit eventual weaknesses.

In order to understand the principles of cryptography, we dive into the terminology that governs our report. The word \emph{cryptography} comes from two Greek words meaning \emph{secret writing} \cite{wikicrypto}. Cryptography is the art and science of concealing meaning, of keeping the enciphered information secret and designing and analyzing encryption schemes. Mathematics is the foundation on which modern encryption rests. Cryptanalysis is the breaking of encoded data and uses statistical and mathematical approaches. The areas of cryptography and cryptanalysis together are called cryptology \cite{wikicrypto}. The basic component of cryptography is a cryptosystem which is based on substitution and permutation. The former changes characters in the plaintext to produce the ciphertext. The latter is an ordered sequence of elements in a finite set with each character appearing only once. In order to make information secret, we use a cipher – a cryptographic algorithm that converts plaintext (i.e. data in readable form) into ciphertext (i.e. a message in unreadable but protected form). We ought to be in possession of a key to be able to convert back the plaintext, otherwise it would not be possible to discover the original content. The process of making text secret is called encryption, and the reverse process is called decryption.

There are three types of cryptography: symmetric key, asymmetric key and hash functions. For this project, we shall focus mainly on the symmetric, also called \emph{conventional}. It comprises a single key for both encryption and decryption and it is the preferred choice when we need to encrypt large amounts of data. The reason being is because it is faster and less power consuming, mainly on the decryption side. Some of the encryption algorithms that use symmetric key encryption are: DES (Data Encryption Standard),Triple DES, AES (Advanced Encryption Standard), Blowfish, Twofish. One of the shortcomings of symmetric key cryptography is related to transferring secure files because the same key has to be used for both encryption and decryption. The sender must find a secure way to provide the recipient with the key so that the latter is able to decrypt the files. Otherwise, they risk placing the key in the wrong hands, leading to deciphering of the encrypted files. Moreover, when the file transfer environment involves multiple users dispersed around the world, it is highly impossible to distribute the key using this crypto method. 

On the other hand, asymmetric cryptography, also called \emph{public key} or \emph{two-key cryptography}, makes use of two keys: public and private. The public key is used to encrypt the message, whereas the private key is used to decrypt it. Most widely employed asymmetric key algorithms are the Rivest–Shamir–Adleman (RSA) and the Digital Signature Algorithm (DSA). However, the asymmetric method is computationally costly and cannot deal with large quantities of data  \cite{wikikey}. The reasons for its downside are the runtime of employed supporting processes, the more storage space required and the more time needed to move over a link. However, asymmetric key cryptography is more secure when we want to transfer over an insecure channel. Furthermore, it does not have the issue with multiple user key distribution, as long as the private key is kept secret.

A cryptographic hash function takes an arbitrary block of data and returns a fixed-size bit string (a hash value, also called (message) digest) such that any change to the data will change the hash value. Prominent examples among hash functions are: Message Digest 5 (MD5), Secure Hash Algorithm 256 (SHA256), Password-Based Key Derivation Function 2 (PBKDF2).

The choice to be made on which system to use for encryption and decryption depends on the purposes, requirements, ease of distribution and on the functions of each algorithm. Modern applications employ a hybrid versions of both symmetric and asymmetric key cryptography as they have their own advantages. For example, Secure Sockets Layer (SSL) and Secure Shell (SSH) are classified as hybrid cryptosystems. 

If someone wants to break the ciphertext, knowing the algorithm but not the specific cryptographic key, there are three main types of attacks. A good cryptosystem should protect against them \cite{stallings2017}:
\begin{itemize}
\item Ciphertext only attack: the ciphertext is known and the goal is to find the corresponding plaintext. If possible, the key may be found as well.
\end{itemize}
\begin{itemize}
\item Known plaintext attack: both ciphertext and plaintext are known and the goal is to find the key.
\end{itemize}
\begin{itemize}
\item Chosen plaintext attack: specific plaintexts are enciphered in order to find out the corresponding ciphertexts and the goal is to find the key.
\end{itemize}




