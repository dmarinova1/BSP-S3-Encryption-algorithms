\documentclass[conference,compsoc]{IEEEtran}

% *** CITATION PACKAGES ***
%
\ifCLASSOPTIONcompsoc
  % IEEE Computer Society needs nocompress option
  % requires cite.sty v4.0 or later (November 2003)
  \usepackage[nocompress]{cite}
\else
  % normal IEEE
  \usepackage{cite}
\fi
% cite.sty was written by Donald Arseneau
% V1.6 and later of IEEEtran pre-defines the format of the cite.sty package
% \cite{} output to follow that of the IEEE. Loading the cite package will
% result in citation numbers being automatically sorted and properly
% "compressed/ranged". e.g., [1], [9], [2], [7], [5], [6] without using
% cite.sty will become [1], [2], [5]--[7], [9] using cite.sty. cite.sty's
% \cite will automatically add leading space, if needed. Use cite.sty's
% noadjust option (cite.sty V3.8 and later) if you want to turn this off
% such as if a citation ever needs to be enclosed in parenthesis.
% cite.sty is already installed on most LaTeX systems. Be sure and use
% version 5.0 (2009-03-20) and later if using hyperref.sty.
% The latest version can be obtained at:
% http://www.ctan.org/pkg/cite
% The documentation is contained in the cite.sty file itself.
%
% Note that some packages require special options to format as the Computer
% Society requires. In particular, Computer Society papers do not use
% compressed citation ranges as is done in typical IEEE papers
% (e.g., [1]-[4]). Instead, they list every citation separately in order
% (e.g., [1], [2], [3], [4]). To get the latter we need to load the cite
% package with the nocompress option which is supported by cite.sty v4.0
% and later.

% *** GRAPHICS RELATED PACKAGES ***
%
\ifCLASSINFOpdf
  % \usepackage[pdftex]{graphicx}
  % declare the path(s) where your graphic files are
  % \graphicspath{{../pdf/}{../jpeg/}}
  % and their extensions so you won't have to specify these with
  % every instance of \includegraphics
  % \DeclareGraphicsExtensions{.pdf,.jpeg,.png}
\else
  % or other class option (dvipsone, dvipdf, if not using dvips). graphicx
  % will default to the driver specified in the system graphics.cfg if no
  % driver is specified.
  % \usepackage[dvips]{graphicx}
  % declare the path(s) where your graphic files are
  % \graphicspath{{../eps/}}
  % and their extensions so you won't have to specify these with
  % every instance of \includegraphics
  % \DeclareGraphicsExtensions{.eps}
\fi
% graphicx was written by David Carlisle and Sebastian Rahtz. It is
% required if you want graphics, photos, etc. graphicx.sty is already
% installed on most LaTeX systems. The latest version and documentation
% can be obtained at: 
% http://www.ctan.org/pkg/graphicx
% Another good source of documentation is "Using Imported Graphics in
% LaTeX2e" by Keith Reckdahl which can be found at:
% http://www.ctan.org/pkg/epslatex
%
% latex, and pdflatex in dvi mode, support graphics in encapsulated
% postscript (.eps) format. pdflatex in pdf mode supports graphics
% in .pdf, .jpeg, .png and .mps (metapost) formats. Users should ensure
% that all non-photo figures use a vector format (.eps, .pdf, .mps) and
% not a bitmapped formats (.jpeg, .png). The IEEE frowns on bitmapped formats
% which can result in "jaggedy"/blurry rendering of lines and letters as
% well as large increases in file sizes.
%
% You can find documentation about the pdfTeX application at:
% http://www.tug.org/applications/pdftex


% *** MATH PACKAGES ***
%
\usepackage{amsmath}
% A popular package from the American Mathematical Society that provides
% many useful and powerful commands for dealing with mathematics.
%
% Note that the amsmath package sets \interdisplaylinepenalty to 10000
% thus preventing page breaks from occurring within multiline equations. Use:
%\interdisplaylinepenalty=2500
% after loading amsmath to restore such page breaks as IEEEtran.cls normally
% does. amsmath.sty is already installed on most LaTeX systems. The latest
% version and documentation can be obtained at:
% http://www.ctan.org/pkg/amsmath

% *** SPECIALIZED LIST PACKAGES ***
%
%\usepackage{algorithmic}
% algorithmic.sty was written by Peter Williams and Rogerio Brito.
% This package provides an algorithmic environment fo describing algorithms.
% You can use the algorithmic environment in-text or within a figure
% environment to provide for a floating algorithm. Do NOT use the algorithm
% floating environment provided by algorithm.sty (by the same authors) or
% algorithm2e.sty (by Christophe Fiorio) as the IEEE does not use dedicated
% algorithm float types and packages that provide these will not provide
% correct IEEE style captions. The latest version and documentation of
% algorithmic.sty can be obtained at:
% http://www.ctan.org/pkg/algorithms
% Also of interest may be the (relatively newer and more customizable)
% algorithmicx.sty package by Szasz Janos:
% http://www.ctan.org/pkg/algorithmicx


% *** ALIGNMENT PACKAGES ***
%
%\usepackage{array}
% Frank Mittelbach's and David Carlisle's array.sty patches and improves
% the standard LaTeX2e array and tabular environments to provide better
% appearance and additional user controls. As the default LaTeX2e table
% generation code is lacking to the point of almost being broken with
% respect to the quality of the end results, all users are strongly
% advised to use an enhanced (at the very least that provided by array.sty)
% set of table tools. array.sty is already installed on most systems. The
% latest version and documentation can be obtained at:
% http://www.ctan.org/pkg/array

% IEEEtran contains the IEEEeqnarray family of commands that can be used to
% generate multiline equations as well as matrices, tables, etc., of high
% quality.

% *** SUBFIGURE PACKAGES ***
%\ifCLASSOPTIONcompsoc
%  \usepackage[caption=false,font=footnotesize,labelfont=sf,textfont=sf]{subfig}
%\else
%  \usepackage[caption=false,font=footnotesize]{subfig}
%\fi
% subfig.sty, written by Steven Douglas Cochran, is the modern replacement
% for subfigure.sty, the latter of which is no longer maintained and is
% incompatible with some LaTeX packages including fixltx2e. However,
% subfig.sty requires and automatically loads Axel Sommerfeldt's caption.sty
% which will override IEEEtran.cls' handling of captions and this will result
% in non-IEEE style figure/table captions. To prevent this problem, be sure
% and invoke subfig.sty's "caption=false" package option (available since
% subfig.sty version 1.3, 2005/06/28) as this is will preserve IEEEtran.cls
% handling of captions.
% Note that the Computer Society format requires a sans serif font rather
% than the serif font used in traditional IEEE formatting and thus the need
% to invoke different subfig.sty package options depending on whether
% compsoc mode has been enabled.
%
% The latest version and documentation of subfig.sty can be obtained at:
% http://www.ctan.org/pkg/subfig

% *** FLOAT PACKAGES ***
%
%\usepackage{fixltx2e}
% fixltx2e, the successor to the earlier fix2col.sty, was written by
% Frank Mittelbach and David Carlisle. This package corrects a few problems
% in the LaTeX2e kernel, the most notable of which is that in current
% LaTeX2e releases, the ordering of single and double column floats is not
% guaranteed to be preserved. Thus, an unpatched LaTeX2e can allow a
% single column figure to be placed prior to an earlier double column
% figure.
% Be aware that LaTeX2e kernels dated 2015 and later have fixltx2e.sty's
% corrections already built into the system in which case a warning will
% be issued if an attempt is made to load fixltx2e.sty as it is no longer
% needed.
% The latest version and documentation can be found at:
% http://www.ctan.org/pkg/fixltx2e

%\usepackage{stfloats}
% stfloats.sty was written by Sigitas Tolusis. This package gives LaTeX2e
% the ability to do double column floats at the bottom of the page as well
% as the top. (e.g., "\begin{figure*}[!b]" is not normally possible in
% LaTeX2e). It also provides a command:
%\fnbelowfloat
% to enable the placement of footnotes below bottom floats (the standard
% LaTeX2e kernel puts them above bottom floats). This is an invasive package
% which rewrites many portions of the LaTeX2e float routines. It may not work
% with other packages that modify the LaTeX2e float routines. The latest
% version and documentation can be obtained at:
% http://www.ctan.org/pkg/stfloats
% Do not use the stfloats baselinefloat ability as the IEEE does not allow
% \baselineskip to stretch. Authors submitting work to the IEEE should note
% that the IEEE rarely uses double column equations and that authors should try
% to avoid such use. Do not be tempted to use the cuted.sty or midfloat.sty
% packages (also by Sigitas Tolusis) as the IEEE does not format its papers in
% such ways.
% Do not attempt to use stfloats with fixltx2e as they are incompatible.
% Instead, use Morten Hogholm'a dblfloatfix which combines the features
% of both fixltx2e and stfloats:
%
% \usepackage{dblfloatfix}
% The latest version can be found at:
% http://www.ctan.org/pkg/dblfloatfix

% *** PDF, URL AND HYPERLINK PACKAGES ***
%
%\usepackage{url}
% url.sty was written by Donald Arseneau. It provides better support for
% handling and breaking URLs. url.sty is already installed on most LaTeX
% systems. The latest version and documentation can be obtained at:
% http://www.ctan.org/pkg/url
% Basically, \url{my_url_here}.

% *** Do not adjust lengths that control margins, column widths, etc. ***
% *** Do not use packages that alter fonts (such as pslatex).         ***
% There should be no need to do such things with IEEEtran.cls V1.6 and later.
% (Unless specifically asked to do so by the journal or conference you plan
% to submit to, of course. )

% correct bad hyphenation here
\hyphenation{op-tical net-works semi-conduc-tor}

\usepackage{hyperref}

\begin{document}
%
% paper title
% Titles are generally capitalized except for words such as a, an, and, as,
% at, but, by, for, in, nor, of, on, or, the, to and up, which are usually
% not capitalized unless they are the first or last word of the title.
% Linebreaks \\ can be used within to get better formatting as desired.
% Do not put math or special symbols in the title.
\title{Encryption Algorithms for Secure Communication over the Internet}

 
% author names and affiliations
% use a multiple column layout for up to three different
% affiliations
\author{\IEEEauthorblockN{BICS student: Desislava MARINOVA\IEEEauthorrefmark{1},
PhD Student Asya MITSEVA\IEEEauthorrefmark{2}, Prof. Dr. Thomas ENGEL \IEEEauthorrefmark{3}}
\IEEEauthorblockA{Faculty of Science, Technology and Communication,
University of Luxembourg,\\
Luxembourg\\
\\
Email: \IEEEauthorrefmark{1}desislava.marinova.002@student.uni.lu,
\IEEEauthorrefmark{2}asya.mitseva@uni.lu,
\IEEEauthorrefmark{3}thomas.engel@uni.lu}}


% conference papers do not typically use \thanks and this command
% is locked out in conference mode. If really needed, such as for
% the acknowledgment of grants, issue a \IEEEoverridecommandlockouts
% after \documentclass

% for over three affiliations, or if they all won't fit within the width
% of the page (and note that there is less available width in this regard for
% compsoc conferences compared to traditional conferences), use this
% alternative format:
% 
%\author{\IEEEauthorblockN{Michael Shell\IEEEauthorrefmark{1},
%Homer Simpson\IEEEauthorrefmark{2},
%James Kirk\IEEEauthorrefmark{3}, 
%Montgomery Scott\IEEEauthorrefmark{3} and
%Eldon Tyrell\IEEEauthorrefmark{4}}
%\IEEEauthorblockA{\IEEEauthorrefmark{1}School of Electrical and Computer Engineering\\
%Georgia Institute of Technology,
%Atlanta, Georgia 30332--0250\\ Email: see http://www.michaelshell.org/contact.html}
%\IEEEauthorblockA{\IEEEauthorrefmark{2}Twentieth Century Fox, Springfield, USA\\
%Email: homer@thesimpsons.com}
%\IEEEauthorblockA{\IEEEauthorrefmark{3}Starfleet Academy, San Francisco, California 96678-2391\\
%Telephone: (800) 555--1212, Fax: (888) 555--1212}
%\IEEEauthorblockA{\IEEEauthorrefmark{4}Tyrell Inc., 123 Replicant Street, Los Angeles, California 90210--4321}}




% use for special paper notices
%\IEEEspecialpapernotice{(Invited Paper)}




% make the title area
\maketitle

% As a general rule, do not put math, special symbols or citations
% in the abstract
\begin{abstract}

Cryptography is a broad method for secure data disguise and transmission. The primary purpose of this paper is to explore the state-of-the-art of cryptographic primitives, both in theory and in practice. The project mainly focuses on symmetric-key cryptography and reviews the types of ciphers it encompasses: transposition and substitution along with block and stream. The project carries on with an implementation of a chat application, in Python programming language, that encrypts messages before transmitting them over the Internet. The encryption algorithms used to develop the client/server infrastructure are: Caesar cipher, ROT13, DES and AES. The latter two have been implemented through the means of PyCryptodome library. Being able to choose the type of encryption, once connected to the server, the client can easily compare the efficiency of the implemented algorithms. As a result, the basic features, advantages, drawbacks and applications of various symmetric-key cryptography algorithms have been mentioned in this paper.


\end{abstract}

% no keywords

% For peer review papers, you can put extra information on the cover
% page as needed:
% \ifCLASSOPTIONpeerreview
% \begin{center} \bfseries EDICS Category: 3-BBND \end{center}
% \fi
%
% For peerreview papers, this IEEEtran command inserts a page break and
% creates the second title. It will be ignored for other modes.
\IEEEpeerreviewmaketitle

\section{Introduction ($\pm$ 5\% total words)}
% no \IEEEPARstart
We are living in the information age enabled by computing and communication technologies whose rapid evolution is almost taken for granted today. This is an age dense with electronic connectivity, eavesdropping and fraud. We send messages around the world instantaneously and we do not want them to be intercepted or stolen. Thus, information is an asset that has precious value like any other. Society’s reliance on a changing panoply of information technologies and technology-enabled services, the increasing global nature of commerce and business, and the ongoing desire to protect freedoms all suggest that future needs for data security is vital.  Network – based attacks spring up every day, distance does not matter anymore since data has been digitalized and there is the issue of privacy being collected, processed and misrepresented. For this reason, the disciplines of cryptography and network security have matured. Thus, internet security is complex yet fascinating. To develop a security mechanism (algorithm or protocol) one must always consider potential attacks, take countermeasures, act like an intruder and go against one’s own intuition and exploit eventual weaknesses in the algorithm. 

We dive into the terminology that governs our topic in the next paragraph. The word “cryptography” comes from two Greek words meaning “secret writing” [1]. Cryptography is a vast, rich subject. It is the art and science of concealing meaning, of keeping the enciphered information secret and of designing and analyzing encryption schemes. It involves mathematics as a supporting tool. Cryptanalysis is the breaking of codes and uses statistical and mathematical approaches. The areas of cryptography and cryptanalysis together are called cryptology. The basic component of cryptography is a cryptosystem which is based on substitution (it changes characters in the plaintext to produce the ciphertext) and permutation (an ordered sequence of elements in a finite set with each character appearing only once).
If someone wants to break the ciphertext, knowing the algorithm but not the specific cryptographic key,  there are 3 main types of attacks [Ref: Stallings, 7th ed, p.90]: 
-	ciphertext only attack – ciphertext known, the goal is to find the corresponding plaintext. The key may be found as well if possible. (A Caesar cipher is susceptible to a statistical ciphertext-only attack)
-	known plaintext attack – both ciphertext and plaintext and known; the goal is to find the key
-	chosen plaintext attack – specific plaintexts are enciphered in order to find out the corresponding ciphertexts and the goal is to find the key.
	a good cryptosystem protects against all three types of attacks. 

We often stumble upon impending security threats such as information leakage; integrity violation; denial of service; illegitimate usage; trojan horses; insider attacks; problems related to access or authentication control. 
Thus, there are instilled security requirements of utmost importance. We present them as follows: confidentiality; integrity; authentication; availability; access control; non-repudiation, which are often intermingled. Hence, one of the mechanisms promising to ensure security is encryption, along with digital signatures and hashing. 

In order to make information secret, we use a cipher – an algorithm that converts plain text into ciphertext, which is in unreadable form, unless we have a key that lets us convert back the cipher. The process of making text secret is called encryption, and the reverse process is called decryption.

There are two types of encryption: symmetric and asymmetric. For this project, we shall focus mainly on the symmetric, also called “conventional”. It comprises a single key for both encryption and decryption and it is the preferred choice when we need to encrypt large amounts of data. On the other hand, asymmetric cryptography, also called “public key” and “two-key”, consists of two keys: public and private. The public key is used to encrypt the message, whereas the private key is used to reverse the encryption. This method cannot deal with large quantities of data due to the runtime of supporting processes employed. However, it is more secure when we want to transfer over an insecure channel. Nevertheless, the choice to be made on which system to use for encryption and decryption depends on the purposes and on each algorithms’ functions. 


\section{Project description ($\pm$ 10\% total words) }
\subsection{Domain}
The report draws on a variety of disciplines. However, it is impossible to appreciate the vastness and significance of the topic in a limited amount of content pages. Nevertheless, we attempt at making the report self-contained by providing the reader with an intuitive understanding of our survey and application results.
The domains that are associated with our project are: 
-	Internet security: "measures to protect data during its transmission over a collection of interconnected networks" [Stallings, ed.3,]
-	Computer Security: "generic name for the collection of tools designed to protect data and to thwart hackers and malicious software (viruses) " [Stallings, ed.3,]; at its heart there are three key objectives: confidentiality, integrity and availability
-	Cryptographic primitives
-	Symmetric-key cryptography
-	cryptographic algorithms: "study of techniques for ensuring the secrecy and/or authenticity of information => 3 main studies of this category (1) symmetric encryption, (2) asymmetric encryption and (3) cryptographic hash functions" [Stallings, 5th ed, p. 3]



\subsection{Objectives}
The objectives of this project are three-fold. It We aim for a comprehensive survey of both principles and practice of cryptography and network security. In a first part, we address a highly needed background material.  Our main focus falls on symmetric encryption, including classical and modern algorithms. We give a basic introduction to the concepts of block ciphers and stream ciphers and how they obtain their security. We put emphasis on DES, AES (+ other??) algorithms. Then, we implement an encrypted chat system that could be of use to provide network security.
\subsection{Constraints}
Our project takes into account substitution/transposition; block/stream ciphers but we do not consider:
-	Asymmetric ciphers (public-key algorithms, including RSA and elliptic curve); 
-	Data integrity algorithms (cryptographic hash functions (message authentication codes (MACs); digital signatures)); 
-	Authentication techniques: key management and key distribution topics (protocols for key exchange)



\section{Background ($\pm$ 15\% total words)}
\subsection{Scientific}

In this section we present substitution and transposition ciphers along with stream and block ciphers. On the one hand, substitution ciphers they take the i-th element of a string and change it for another value to produce a ciphertext. Substitution ciphers can be monoalphabetic or polyalphabetic. The former, also called simple substitution cipher, maps a plaintext letter to a ciphertext letter based on a single alphabet key. The latter uses multiple substitution alphabets. In transposition ciphers the values of plain letters are not changed for another but rearranged. Modern ciphers use a combination of both substitution and transposition ciphers. Ciphers can also be divided into stream and block ciphers. The former convert one symbol (1 bit or byte) of plaintext directly into a symbol of ciphertext. The latter encrypt a group of plaintext symbols as one block and outputs a block of ciphertext. We add that substitution ciphers can operate on either blocks or streams. 

In the rest of the section, we describe different examples of ciphers for each of the groups presented above.

\textbf{Substitution ciphers:} 
The most popular example of substitution cipher is the One - Time Pad (OTP), designed in 1917. What makes it very secure is that a pad (secret key) is never reused. Instead, OTP utilizes a random key which has an equal length to the message. It shifts each plain letter depending on the corresponding keyword letter. The random key is also used in encryption and decryption of the message, and afterwards is discarded \cite{stallings2011}. If we want to generate a new message, a new key is required. Thus, this scheme leads to an output unrelated to the plaintext. If we try to decipher a ciphertext using this method, and let us suppose we have succeeded in finding the key, our attempts would lead us to fairly different decrypted outputs. For this reason, it would be difficult to decide upon the correct decryption. However, there are two fundamental complexities related to the one-time pad: (1) making large quantities of random keys; (2) key distribution and protection (since a key of equal length is needed by both sender and receiver)\cite{stallings2011}. Hence, if we ought to employ the one-time pad, it would be best for low-bandwidth channels requiring very high security \cite{stallings2011}. Nevertheless, this cryptosystem is regarded as the sole exhibiting perfect secrecy \cite{stallings2011}. Some cipher machines attempted mechanically then electronically, in both cases nevertheless unsuccessfully, to create approximations to one-time pads (OTPs). Many snake oil algorithms claim to be unbreakable, posing as OTPs. Hence, these algorithms give rise to pseudo-OTPs providing pseudo-security \cite{curtin1998}.

How does a substitution cipher work? In order to produce the ciphertext, the cipher modifies characters in the plaintext. Can it be broken? We can break the monoalphabetic cipher by using letter frequency analysis since letter frequencies are preserved (e.g. 'e is the most common letter in English, followed by 't,i,o,a,n,s,r '). The polyalphabetic cipher can be broken by decomposing into individual alphabets and as a consequence to treat it as a simple substitution cipher. A historical example of breaking the substitution cipher is the execution of Mary, Queen of Scots, in 1587 for plotting to kill Queen Elizabeth \cite{maryscots}. The "Vigenere Cipher" is the most popular polyalphabetic cipher. The Vigenere cipher is an improvement of the Caesar cipher but not as secure as the unbreakable One Time Pad. Caesar cipher encodes each plain letter by a constant shift, whereas the One Time Pad shifts each plain letter depending on the corresponding keyword letter. The Vigenere cipher uses a keyword of given length repeatedly to determine the encoding shift of each plain letter. For instance, we are given a word, we convert it to numbers, according to the letter position in the alphabet. Next, we repeat the sequence of numbers along the message. Then, we encrypt each letter in the message by shifting, according to the number below it, and send the encrypted message openly to the recipient. Hence, we employ multiple shifts. To decrypt the message, we should subtract the shifts according to the secret word or key we have a copy of.

\textit{Cipher machines}: Cryptography was mechanized by the 1900s in the form of encryption machines. The basic component of a cipher machine is the wired rotor. When we get prior to the Second World War, the Germans realized that, thanks to radios, messages can be sent across the battlefield in an instant but that always meant the other side could also tap into those radio channels. Therefore high-tech encryption was highly needed. The Germans invented Enigma,whose complexity guaranteed their privacy. The Enigma was a substitution cipher, but a more sophisticated one because it used three rotors in a row, each feeding into the next. The Germans typed their messages on a keyboard that came out as gibberish on a lampboard. Then they sent it over the radio to the other side, which also has the same Enigma machine to help them decrypt the unintelligible message. Furthermore, to decrypt the message, it is assumed that the other side should know the algorithm and must have configured the machine the same way as the Enigma machine encrypting the message. The Enigma relies on a random letter generator. Hence, its encryption does not seem to follow any kind of pattern. There are 26 wires coming out of the keyboard, running through three rotors with 6 permutations, going into the lamps representing the output letter. Once the first of the three rotors hits a full evolution, it kicks the next rotor to start moving, yet again, when it completes its cycle, it transmits the process to the last rotor. The inside of the rotors resembles scrambled wiring. A rotor's side has 26 junctions or contacts, accepting the incoming wires and outputting them to the other side of the rotor. The rotor moves each time a letter is typed in. Even if the same letter is passed sequentially, the rotor would move and a lamp representing a different output letter would light up. Thus, the dynamism of the rotors accounts for the complicated encryption. Finally, a plugboard at the front of the machine allows letters to be optionally swapped so that the machine it is interacting with is configured the same way. Alan Turing and his colleagues at Bletchley Park were able to break the Enigma codes and automate the process by developing a new machine called \textit{the Bombe} \cite{alanturing}. The rotor machines were cracked because the same key had been used over an extended period of time and due to the use of old compromised keys while encrypting. In addition, the circuit's configuration showed that it was impossible for a letter to be encrypted as itself, which turned out to be a cryptographic flaw. 

\textbf{Transposition ciphers:} Another kind of mapping is achieved by performing permutation on the plaintext letters. This technique is referred to as a transposition cipher. For example, a simple cipher of this sort is the rail fence technique. The plaintext is written as a sequence of diagonals and read off as a sequence of rows. The key is the number of rows used to encode. 

With today's computer power transposition ciphers can be broken quickly. Trying to compute the frequencies of the cipher letters is one method. Another way is testing possible rearrangements. For instance, one may try to read the cipher text backwards. If that does not yield the plain text then the rail fence technique could be tested. If that does not yield the plain text, one could check if two consecutive letters were switched.    

\textbf{Stream ciphers:} A stream cipher generates a stream of bytes, one for each byte of the text we want to encrypt.  An example is Rivest Cipher 4 (RC4), designed in 1987 by Ron Rivest. RC4 uses either 64-bit or 128-bit key sizes. Its most popular implementation is in WEP for 802.11 wireless networks and in SSL. RC4 consists of a key-scheduling algorithm, which initializes a permutation of all 256 possible bytes (since the permutation array has a length of 256 bytes) \cite{stallings2017}. The permutation itself is started with a length key between 40 and 2048 bits. RC4 is also composed of pseudo-random generation algorithm (PRGA) which generates the stream of bits. PRGA has two 8-bit index pointers (i and j) on which, during the 256 iterations, it executes operations such as XOR-ing, swapping, modulo. Stream ciphers are vulnerable to attacks, mostly to bit-flipping \cite{wikirc4} . Bit-flipping attack deals with changing a bit in a ciphertext so that it results in a predictable plaintext. It is not targeting the cipher itself but a message or a series of messages on a channel. For this reason, it could turn into a Denial of Service attack. We should never reuse a key with a stream cipher. The main reason being that we can recover the plaintext, using the keystream and the ciphertext. The keystream could be recovered as well using the plaintext and the ciphertext. Furthermore, if we use two ciphertexts from the same keystream, we can recover the XOR-encryption of the plaintexts.

\textbf{Block ciphers:} Block ciphers date back to late 1960s when IBM attempted to develop banking security systems \cite{ibmcrypto}. The result was Lucifer, an encryption method, with 128-bit key and block size, aimed at protecting data for cash-dispensing system in the UK. However, it was not secure in any of its version implementations. 
When we use block ciphers, each block is encrypted independently producing a ciphertext block of equal size. The block size can be 64-bits, 128-bits or 256-bits. The ciphertext is generated from the plaintext and the key by iterating a \emph{round function} (as we go through it several times). Input to the round functions consists of key and the output of previous round. Block ciphers are also known as \emph{product} ciphers and are built with a \emph{Feistel structure}, as first described by Horst Feistel of IBM in 1973. The plaintext is split into left and right halves. The Feistel design consists of a number of identical rounds of processing. During each round, substitution is performed on one half of the processed data, followed by a permutation that interchanges the two halves. What is more, the original key is expanded so that a different key is used for each round. Symmetric block encryption algorithms are based on this structure. In general, it is accepted that block ciphers are applicable to a broader range of applications in software as opposed to stream ciphers.

When we want to encrypt longer than a 16-byte plaintext (email, file, etc.) with a symmetric key block cipher algorithm, there exist several modes of operation:
\begin{itemize}
\item ECB (Electronic Code Book mode)
\end{itemize}
\begin{itemize}
\item CBC (Cipher Block Chaining mode)
\end{itemize}
\begin{itemize}
\item CFB (Cipher Feedback mode) 
\end{itemize}
\begin{itemize}
\item OFB (Output Feedback mode)
\end{itemize}
\begin{itemize}
\item CTR (Counter mode)
\end{itemize}
These modes aim at providing confidentiality and protection for sensitive data. The latter three modes use the block cipher as a building block for a stream cipher \cite{springer2010}. To encrypt data with DES and AES our project utilizes ECB and CBC modes, respectively. ECB requires that the length of the plaintext is a multiple of the block size of the cipher used. If the plaintext does not conform to the required length, it must be padded. Padding is achieved by appending as many zero bits as possible in order to reach a multiple of the block length (i.e. 8, 16, 32, etc.). In case the plaintext conforms to the length, an extra block is appended consisting of only padding bits. Furthermore, in ECB mode each block is encrypted separately. One of the advantages of ECB is that if a transmission problem occurs, the received encrypted block could still be decrypted \cite{springer2010}. Moreover, parallelization is allowed in ECB mode, i.e. one encryption unit encrypts block 1, the next one block 2, etc., which is useful in high-speed implementations \cite{springer2010}. Nevertheless, there are some weaknesses associated with the ECB mode. During encryption identical plaintext blocks result in identical ciphertext blocks as long as the key does not change \cite{springer2010}. Hence, an attacker could easily deduce information. The ECB mode is susceptible to substitution attacks because once block mapping from plaintext to ciphertext is known, a sequence of ciphertexts can be manipulated \cite{springer2010}. 
With regard to CBC, all encrypted blocks are chained together such that a certain ciphertext depends not only on a certain block but on all previous plaintext blocks as well \cite{morris2001}. Each plaintext block also gets XORed with the previous ciphertext block prior to encryption \cite{morris2001}. In addition, the encryption is randomized by using an initialization vector (IV). In our case, the IV is new every time we encrypt and is always incremented when a new session starts. If we encrypt a string once with a first IV and a second time with a different IV, the two resulting ciphertext sequences look completely unrelated to each other. To strengthen the encryption method, a mode of operation is used, for AES we have chosen CBC mode. 

\textit{DES}: Cryptography gradually moved from hardware to software with the advent of computers. Most famous example of the block cipher design and the classic Feistel structure is the Data Encryption Standard (DES), designed by IBM under the advisement of NASA in 1977 and standardized 2 years later. It was meant to encipher sensitive but non classified data. DES complexity is comprised of a simple repetition of the primitives of transposition, substitution, split, concatenation and bit-wise operation. DES uses a 56-bit key. The cipher is thoroughly examined in section 4 of this report for it has been used during our hands-on approach.

\textit{TRIPLE DES (3DES)}: has replaced DES since the simpler version is susceptible to brute force attacks. 3DES is a more secure method of symmetric- key encryption, as it encrypts data three times in contrast to DES, i.e. the one 56-bit key becomes three individual keys rendering a 168-bit key. However, initiating three instances of DES, implies that 3DES is much slower than other methods of encryption. The text is encrypted firstly with key 1, then decrypted with key 2 and lastly encrypted once more by key 3. Triple DES offers a security level of $2^{112}$ instead of $2^{168}$,i.e., with only two keys of encryption, since the 168-bit key can be cumbersome to implement. This method is known as Encrypt-Decrypt-Encrypt (EDE): key 1 encrypts the message; then the message is decrypted using key 2, afterwards the text is encrypted again with key 2. We should mention that there exists double DES as well, which is composed of two successive instances of DES. 2DES offers a security level of $2^{57}$ instead of $2^{112}$ due to the cryptographic attack, called \emph{meet-in-the-middle} \cite{mitm}. If we take some plaintexts and encode them and at the same time take some encrypted values and start decrypting them, we only have to look for where they meet in the middle with the same value. Those intersections then reveal the key. 3DES avoids this as we would need to perform a third operation to tell if they met in the middle. Thus, it is not enough to look for where the first and last operation produce the same value.

\textit{AES}: The Advanced Encryption Standard (AES) replaced DES in 2000 as the US Government encryption technique to protect classified information. The symmetric block cipher was developed by two Belgian cryptographers Joan Daemen and Vincent Rijmen. AES was designed to be efficient in both hardware and software. It supports a block length of 128 bits and key lengths of 128, 192, and 256 bits. Thus, brute-force attacks are much harder to be launched against it. AES chops data up into 16-byte blocks, transferred to a \emph{state array}, and then applies a series of substitutions (e.g. \emph{substitute bytes}, \emph{MixColumns}) and permutations (e.g. \emph{ShiftRows}), based on the key value. We note that the state array undergoes various modifications throughout both encryption and decryption. Moreover, that way, it obscures the message, by adding diffusion, confusion and non-linearity and repeating the processes ten (for a 16-byte key) or more times (12 rounds for 24-byte key; 14 rounds for a 32-byte key) for each block. In order to \emph{substitute bytes}, an S-box is used to perform a byte-by-byte substitution of the block. The other substitution, called \emph{MixColumns}, utilizes arithmetic over ($2^{8}$) \cite{stallings2017-a}. In fact, all encryption algorithms necessitate arithmetic operations. The key is expanded into an array of key schedule four-byte words. Employing a bitwise XOR of the current block with part of the expanded key is a stage called \emph{AddRoundKey}, solely used on the key. Therefore, the cipher is locked around this stage adding to the security of the AES encryption.  At the end, the state of the plaintext is copied to an output matrix where the bytes are ordered in a column. Similarly, the bytes of the expanded key, which form a word, are placed in the column of the matrix. Today, AES is used everywhere, from encrypting files and sensitive data, transmitting data over WiFi with WPA2, to accessing websites using HTTPS. 

\subsection{Technical}

One of the deliverables of this project is a chat system implemented in Python. The application employs ciphers reviewed in this report to encrypt messages. Python is a programming language created by Guido van Rossum in late 1980s in the Netherlands. Python is a free software and the latest version can be downloaded from \url{www.python.org}. Python is a simple and very powerful general purpose computer programming language. This allows for the language to remain fresh and current with the newest trends. Python has libraries for just about everything. It can be used for web development, web scraping, writing scripts, browser automation, GUI development, data analysis, machine learning, computer vision, and game development, etc.. In addition, Python is an object-oriented programming language (OOP). There are four pillars of OOP: encapsulation, abstraction, inheritance, and polymorphism. Before turning to OOP, there is procedural programming, a simple and straightforward programming, that divides a program into a set of functions. However, as our program grows we end up with \emph{spaghetti code}, i.e. many functions all over the place that are interdependent. Thus, OOP offers a solution to this issue. We can bundle a group of related variables (referred to as \emph{properties}) and functions that operate on them (referred to as \emph{methods}) into an object. This grouping is called \emph{encapsulation}. Using this technique, we can reduce complexity and increase reusability. In Python, objects are data and have a certain type (integer, float, list, etc.). Once we have created our objects we can manipulate and interact with them (append, sort, delete, concatenate, etc.). We can hide the details and complexity (e.g. some methods and properties) from the outside and show only the essentials through the process of \emph{abstraction}. This is beneficial as we produce simpler interface and reduce the impact of change (i.e. no inner changes leak to the outside of the contained object). \emph{Inheritance} is a mechanism that allows us to eliminate redundant code. \emph{Polymorphism} is a technique that allows us to refactor long if/ else or switch/case statements. 

Our platform is macOS and we can open Python on a Terminal. Moreover, we can use any Text Editor or IDLE of our choice. For this project, we have used the source code editor Visual Studio Code (VSC) developed by Microsoft. VSC is a lightweight yet powerful editor. It offers support for debugging, embedded Git control, syntax highlighting, snippets, extensions, smart code completion and code refactoring. A GitHub repository was set up to share and build our project. The link is provided in the annex of the report. GitHub is a web-based hosting service of open source projects for version control using Git. 

To build the chat program for our project we need a client and a server. To establish a connection between both and be able to transmit information back and forth over the Internet, we need a socket link. Sockets aid the communication between these two entities. The client requests information from the server and the server carries out data to the client. Similarly, the client is a program as well. However, altogether, they are referred to as \emph{client/server architecture}. For instance, when we visit a website, we are using a socket and accessing a port of the web server. In this case, the server has generally port 80 open, used to transfer HTTP data. Other websites have ports 21 and 20 open for FTP access, which is not very secure, some have port 22 intended for SSH. The lower number ports are specific ports, whereas the higher number ports signify general purpose rights. More often than not, questions arise regarding security when using higher number ports. 

Our client should send an initial request to the server's port number, \emph{the listening port}. To keep the communication over TCP (valid for every other protocols as well), we have the client's IP address and local port number as well as the server's IP address along with its port number. Each client that connects to the server gets unique set of values of two IP addresses and two port numbers in a tuple, a collection of items. 

\section{BSPro - A First Bachelor Semester Project in BiCS-land}
\subsection{Requirements ($\pm$ 15\% total words)}
[[[[Describe here all the properties that characterize the deliverables you produce. It should describe what are those deliverables, who are the actors exploiting the deliverables, what are the expected functional and non functional qualities of the deliverables.]]]]

Ciphers have been used long before apparition of computers. During Roman times, Julius Caesar invented an encryption for his private correspondence. Today, it is known as the Caesar cipher. Given the English alphabet, the cipher shifts a letter from the alphabet three places to the right, i.e. A is encrypted as D, B as E,etc. It is a monoalphabetic cipher. To decrypt the message, the other party needs to know both the algorithm and the shifting number. The secret key shared by the sender and the recipient of the message is k=3. Breaking the Caesar Cipher can be done by testing all possible shifts. Since an alphabet of length 26 is used we have to test 26 shifts. 

ROT13 is a simple monoalphabetic substitution cipher, a special class of Caesar cipher, that encodes a certain letter with another letter that is 13 positions after it. Only those letters which occur in the English alphabet are affected. Numbers, symbols, whitespace, and all other characters are left unchanged. It is an example of a cipher providing weak encryption, since both operations encryption and decryption are identical. Hence, this cipher is its own inverse. Because we know there are 26 letters in the English alphabet, if we wish to apply twice 13 it would give us one shift of 26. Thus, it leads us back to the original text. Moreover, the direction of the shift is of no importance, since it will always give the same output \cite{swenson2008modern}.
ROT13 is used in online forums as a means of hiding spoilers, punchlines, puzzle solutions, and offensive materials from the casual glance. Furthermore, it has inspired a variety of letter and word games online \cite{wikirot13}. 

Another cipher our project deals with is the block cipher Data Encryption Standard (DES). In DES we put 64-bit block of plaintext,  DES' key (which does the processing) is 64 bit which is 8 bytes but for each byte there is one parity bit, therefore, the value in the key is only 56 bits which means there are 2 to the power of 56 different keys. The output ciphertext is a 64-bit block. 
From the 56-bit key, 16 bits are generated (one for each round). Each DES round works consecutively, has the same operations and uses a different key. Each round uses a combination of proper substitution, where we take some bits which are substituted with another combination of bits. Each DES round takes as an input the ciphertext produced by the previous round and outputs the ciphertext for the next round. The input is divided into a left half and a right half. The output left half is just the right half of the input. The right output is the result of XOR-ing the left half of the input and the output of the Mangler Function (which takes as an input the 32-bit right half, expands it to 48-bit (bit-wise operation) then XORs it to 48-bit key, then uses the S-Boxes to substitute the 48-bit value into a 32-bit value).
The algorithm process of decryption in DES is the same as the encryption process. It uses the ciphertext as an input to DES but the keys are run in reversed order, i.e. k = 16 is used as the first round of decryption, k = 15 is used as a second round of decryption and so on, so forth. 
Diffusion is one of the principles in encryption. It is achieved through permutation (initial transposition). Permutation works by changing the position of the bits in DES. The mixed bits are taken to a sub-box which receives the 56-bit key and the 64-bit plaintext, once it completes processing, it outputs the 64 bits into another transposition subsection, which in turn produces a 64-bit ciphertext. 
The larger the block size, the key size and number of rounds means greater security. However, when there are more than 16 rounds or more than 56 keys, the security will neither be increased nor the encryption will be made any stronger. The possible attacks that often occur on product ciphers are differential cryptanalysis and linear cryptanalysis, however, according to Stallings, DES has proven to be resistant to these sort of attacks \cite{stallings2011-a}. By 1999, a computer could try every possible key in a couple of days rendering the cipher insecure.

\subsection{Design ($\pm$ 20\% total words)}

The technical part of our project consists of creating a chat application. Our source code editor is Visual Studio Code and we program in Python. We create a TCP server as well as a TCP client that connects to the server. 
For the client to use TCP Networking, we require a socket module that serves as a communication endpoint with two parameters:
s= socket.socket(socket.AF_INET, socket.SOCK_STREAM)
the first parameter AF_INET stands for IPv4, the second parameter \textit{SOCK_STREAM} means we are going to use TCP. We specify a variable for port number (on which the server is running) and ip, which holds the ip of the host machine it is running on. We use \textit{gethostbyname} to look for the host’s ip since the host machine does not have a fixed IP address and often when it reconnects to an Internet connection is assigned a new IP address.
\textit{ip = str(socket.gethostbyname(socket.gethostname()))}
\textit{port = 1234}
Once we the port and ip address are known, we can connect to the server by passing the function to a data structure \textit{s.connect(((ip, port))}

In the Python script for the server, we turn the connections into threads creating a multithreaded server. A thread is an external process and we can have many running simultaneously. This facilitates the whole process when there are many clients that connect to the server. We import the thread library import threading. We create an EchoThread class and pass to its function a connection and address. For the server to use TCP Networking, we require a socket module that serves as a communication endpoint with two parameters:
\textit{s= socket.socket(socket.AF_INET, socket.SOCK_STREAM)}
There is also a callback to every connection established to that server with variable “conn”, short for connection. It describes information about the connection. TCP allows us to open a connection, execute some data transfer on it without even closing the connection which may stay open for a long period of time. We subscribe to two events. The first one is data and second one is closing the connection. For the data event, each time data is sent over to the server, data is received by the client. Once the client cuts the connection or the EchoThread is terminated, the connection is ended.
For the client to work, a port number is specified equal to the server’s so that both can connect. When the port and host have been bounded, we can listen to incoming connections with socket variable s and calling listen on it: \textit{s.listen(10)}. We pass the variable 10 which means that up to 10 people can be queued for us to handle the requests. However, the eleventh will be rejected. To accept requests from outside, we use connection variable and address variable as: \textit{conn, addr = s.accept()} which stores the ip import of the client trying to connect and we call accept on the socket we have created. Upon a connection being established, the client can send messages to the server that will cause it to echo back the message. The server can receive data from client by creating variable data and sending it to the response gathered from connection. We write: \textit{data = conn.recv(4096)}. Then server sends that message to all connected clients \textit{conn.sendall(data}. Afterwards we can close the socket connection between client and server \textit{print("connection closed with ", addr)} as well as close the socket itself \textit{s.close()} that allows connections to exist.
For our project, we aim at implementing in Python several encryption and decryption applications. We make use of Caeser Cipher, ROT13, DES and AES.
DES and AES have been implemented through the means of a crypto library “PyCryptodome”. It is a self-contained Python package of low-level cryptographic primitives. It supports PyPy, Python2 and Python3. PyCryptodome can be used as a replacement for the old PyCrypto library. We install all modules under Crypto package with pip3 install pycryptodome.  However, having both PyCrypto and PyCryptodome installed at the same time is not a good idea as they will interfere with each other. If, however, one insists on having them both, it would be best to deploy them in a virtual environment. To have an independent library of the old PyCrypto, it suffices to type the following command in Terminal shell:\textit{pip3 install pycryptodomex} and all modules are installed under Cryptodome package and PyCrypto and PyCryptodome can coexist. 


\subsection{Production ($\pm$ 20\% total words)}

We implement all algorithms with Pythong Programming language. 

Caeser Cipher is a substitution cipher and a type of shift cipher. Shift ciphers work by using the modulo operator to encrypt and decrypt messages. In the following lines we demonstrate its algorithm. We take an alphabetic message (A to Z). We take a key, an integer from 0 to 25, equal to 3.To encrypt, either left-shift or right-shift letter by letter by the value of key. For instance, if message is ABC and key K is 3, we perform right-shift and the encrypted text is A + 3 = C, B+3 = E, C+3 = F, hence, CEF. To decrypt, left-shift or right-shift the message letter by letter opposite to the shift performed in encryption by the value of key. For example if encrypted text is "CEF" and key is 3 and we are performing left-shift then encrypted text will be C-3 =A , E-3 =B B, F-3 = C, hence, ABC being decrypted text. 
For every letter (char) in the message, we find the letter that matches its order in the alphabet starting from 0. We calculate \textit{newPosition = (position + key) mod 26}, then convert \textit{newPosition} into a letter that matches its order in the alphabet starting from 0. The decryption process is similar; however, we subtract the key during the modular operation.

To implement DES, we import DES function from Cryptodome library module. Then we create and specify a \textit{key = “mysecret”}, a password that shall be used to encrypt the text and is 8 bytes long. We have created a pad function taking as input text. Whatever we type for the text parameter, the pad function shall operate on the text. We run a while loop checking the length of text mod 8, i.e. the modulus divides it and shows the remainder. If remainder is not equal to zero, we add a space at the end of text such that it becomes multiple of 8 (e.g. 8, 16, 24, 32, etc.), otherwise the encryption is not going to work as DES encryption algorithm takes input in 8 bytes. For instance, if we type ‘abcd’ it is not going to work, hence, the function adds 4 empty spaces at the end of ‘abcd’ to make sure it is a total of 8 characters/bytes. Then, we create an object, called ‘des’. We use \textit{new()} function of module DES that takes two parameters: key and \textit{DES.MODE_ECB} which is the encryption mode. We have created a variable taking an input from client that shall be encrypted. The \textit{padded_text} variable uses the pad() function on the input to make sure the input provided by client is a multiple of 8 and if necessary adds empty spaces to conform to the multiple condition. Next, we have created an \textit{encrypted_text} variable that uses the \textit{encrypt()} function on the \textit{padded_text} and turns it into an encrypted string. In order to decrypt, we run des submodule and pass decrypt() function on it. 



\subsection{Assessment ($\pm$ 15\% total words)}
Provide any objective elements to assess that your deliverables reached or not the requirements described above. 
\section*{Acknowledgment}
The authors would like to thank the BiCS management and education team for the amazing work done.

\section{Conclusion}
The conclusion goes here.


\newpage 
\section{Appendix}
All images and additional material go there.

A link to the GitHub repository of the project: \url{https://github.com/dmarinova1/BSP-S3-Encryption-algorithms}
% that's all folks


\bibliographystyle{IEEEtran}
\bibliography{References}

% An example of a floating figure using the graphicx package.
% Note that \label must occur AFTER (or within) \caption.
% For figures, \caption should occur after the \includegraphics.
% Note that IEEEtran v1.7 and later has special internal code that
% is designed to preserve the operation of \label within \caption
% even when the captionsoff option is in effect. However, because
% of issues like this, it may be the safest practice to put all your
% \label just after \caption rather than within \caption{}.
%
% Reminder: the "draftcls" or "draftclsnofoot", not "draft", class
% option should be used if it is desired that the figures are to be
% displayed while in draft mode.
%
%\begin{figure}[!t]
%\centering
%\includegraphics[width=2.5in]{myfigure}
% where an .eps filename suffix will be assumed under latex, 
% and a .pdf suffix will be assumed for pdflatex; or what has been declared
% via \DeclareGraphicsExtensions.
%\caption{Simulation results for the network.}
%\label{fig_sim}
%\end{figure}

% Note that the IEEE typically puts floats only at the top, even when this
% results in a large percentage of a column being occupied by floats.


% An example of a double column floating figure using two subfigures.
% (The subfig.sty package must be loaded for this to work.)
% The subfigure \label commands are set within each subfloat command,
% and the \label for the overall figure must come after \caption.
% \hfil is used as a separator to get equal spacing.
% Watch out that the combined width of all the subfigures on a 
% line do not exceed the text width or a line break will occur.
%
%\begin{figure*}[!t]
%\centering
%\subfloat[Case I]{\includegraphics[width=2.5in]{box}%
%\label{fig_first_case}}
%\hfil
%\subfloat[Case II]{\includegraphics[width=2.5in]{box}%
%\label{fig_second_case}}
%\caption{Simulation results for the network.}
%\label{fig_sim}
%\end{figure*}
%
% Note that often IEEE papers with subfigures do not employ subfigure
% captions (using the optional argument to \subfloat[]), but instead will
% reference/describe all of them (a), (b), etc., within the main caption.
% Be aware that for subfig.sty to generate the (a), (b), etc., subfigure
% labels, the optional argument to \subfloat must be present. If a
% subcaption is not desired, just leave its contents blank,
% e.g., \subfloat[].


% An example of a floating table. Note that, for IEEE style tables, the
% \caption command should come BEFORE the table and, given that table
% captions serve much like titles, are usually capitalized except for words
% such as a, an, and, as, at, but, by, for, in, nor, of, on, or, the, to
% and up, which are usually not capitalized unless they are the first or
% last word of the caption. Table text will default to \footnotesize as
% the IEEE normally uses this smaller font for tables.
% The \label must come after \caption as always.
%
%\begin{table}[!t]
%% increase table row spacing, adjust to taste
%\renewcommand{\arraystretch}{1.3}
% if using array.sty, it might be a good idea to tweak the value of
% \extrarowheight as needed to properly center the text within the cells
%\caption{An Example of a Table}
%\label{table_example}
%\centering
%% Some packages, such as MDW tools, offer better commands for making tables
%% than the plain LaTeX2e tabular which is used here.
%\begin{tabular}{|c||c|}
%\hline
%One & Two\\
%\hline
%Three & Four\\
%\hline
%\end{tabular}
%\end{table}


% Note that the IEEE does not put floats in the very first column
% - or typically anywhere on the first page for that matter. Also,
% in-text middle ("here") positioning is typically not used, but it
% is allowed and encouraged for Computer Society conferences (but
% not Computer Society journals). Most IEEE journals/conferences use
% top floats exclusively. 
% Note that, LaTeX2e, unlike IEEE journals/conferences, places
% footnotes above bottom floats. This can be corrected via the
% \fnbelowfloat command of the stfloats package.

% trigger a \newpage just before the given reference
% number - used to balance the columns on the last page
% adjust value as needed - may need to be readjusted if
% the document is modified later
%\IEEEtriggeratref{8}
% The "triggered" command can be changed if desired:
%\IEEEtriggercmd{\enlargethispage{-5in}}

% references section

% can use a bibliography generated by BibTeX as a .bbl file
% BibTeX documentation can be easily obtained at:
% http://mirror.ctan.org/biblio/bibtex/contrib/doc/
% The IEEEtran BibTeX style support page is at:
% http://www.michaelshell.org/tex/ieeetran/bibtex/
%\bibliographystyle{IEEEtran}
% argument is your BibTeX string definitions and bibliography database(s)
%\bibliography{IEEEabrv,../bib/paper}
%
% <OR> manually copy in the resultant .bbl file
% set second argument of \begin to the number of references
% (used to reserve space for the reference number labels box)

\end{document}
