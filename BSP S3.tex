\documentclass[conference,compsoc]{IEEEtran}

% *** CITATION PACKAGES ***
%
\ifCLASSOPTIONcompsoc
  % IEEE Computer Society needs nocompress option
  % requires cite.sty v4.0 or later (November 2003)
  \usepackage[nocompress]{cite}
\else
  % normal IEEE
  \usepackage{cite}
\fi
% cite.sty was written by Donald Arseneau
% V1.6 and later of IEEEtran pre-defines the format of the cite.sty package
% \cite{} output to follow that of the IEEE. Loading the cite package will
% result in citation numbers being automatically sorted and properly
% "compressed/ranged". e.g., [1], [9], [2], [7], [5], [6] without using
% cite.sty will become [1], [2], [5]--[7], [9] using cite.sty. cite.sty's
% \cite will automatically add leading space, if needed. Use cite.sty's
% noadjust option (cite.sty V3.8 and later) if you want to turn this off
% such as if a citation ever needs to be enclosed in parenthesis.
% cite.sty is already installed on most LaTeX systems. Be sure and use
% version 5.0 (2009-03-20) and later if using hyperref.sty.
% The latest version can be obtained at:
% http://www.ctan.org/pkg/cite
% The documentation is contained in the cite.sty file itself.
%
% Note that some packages require special options to format as the Computer
% Society requires. In particular, Computer Society  papers do not use
% compressed citation ranges as is done in typical IEEE papers
% (e.g., [1]-[4]). Instead, they list every citation separately in order
% (e.g., [1], [2], [3], [4]). To get the latter we need to load the cite
% package with the nocompress option which is supported by cite.sty v4.0
% and later.

% *** GRAPHICS RELATED PACKAGES ***
%
\ifCLASSINFOpdf
  % \usepackage[pdftex]{graphicx}
  % declare the path(s) where your graphic files are
  % \graphicspath{{../pdf/}{../jpeg/}}
  % and their extensions so you won't have to specify these with
  % every instance of \includegraphics
  % \DeclareGraphicsExtensions{.pdf,.jpeg,.png}
\else
  % or other class option (dvipsone, dvipdf, if not using dvips). graphicx
  % will default to the driver specified in the system graphics.cfg if no
  % driver is specified.
  % \usepackage[dvips]{graphicx}
  % declare the path(s) where your graphic files are
  % \graphicspath{{../eps/}}
  % and their extensions so you won't have to specify these with
  % every instance of \includegraphics
  % \DeclareGraphicsExtensions{.eps}
\fi
% graphicx was written by David Carlisle and Sebastian Rahtz. It is
% required if you want graphics, photos, etc. graphicx.sty is already
% installed on most LaTeX systems. The latest version and documentation
% can be obtained at: 
% http://www.ctan.org/pkg/graphicx
% Another good source of documentation is "Using Imported Graphics in
% LaTeX2e" by Keith Reckdahl which can be found at:
% http://www.ctan.org/pkg/epslatex
%
% latex, and pdflatex in dvi mode, support graphics in encapsulated
% postscript (.eps) format. pdflatex in pdf mode supports graphics
% in .pdf, .jpeg, .png and .mps (metapost) formats. Users should ensure
% that all non-photo figures use a vector format (.eps, .pdf, .mps) and
% not a bitmapped formats (.jpeg, .png). The IEEE frowns on bitmapped formats
% which can result in "jaggedy"/blurry rendering of lines and letters as
% well as large increases in file sizes.
%
% You can find documentation about the pdfTeX application at:
% http://www.tug.org/applications/pdftex


% *** MATH PACKAGES ***
%
%\usepackage{amsmath}
% A popular package from the American Mathematical Society that provides
% many useful and powerful commands for dealing with mathematics.
%
% Note that the amsmath package sets \interdisplaylinepenalty to 10000
% thus preventing page breaks from occurring within multiline equations. Use:
%\interdisplaylinepenalty=2500
% after loading amsmath to restore such page breaks as IEEEtran.cls normally
% does. amsmath.sty is already installed on most LaTeX systems. The latest
% version and documentation can be obtained at:
% http://www.ctan.org/pkg/amsmath

% *** SPECIALIZED LIST PACKAGES ***
%
%\usepackage{algorithmic}
% algorithmic.sty was written by Peter Williams and Rogerio Brito.
% This package provides an algorithmic environment fo describing algorithms.
% You can use the algorithmic environment in-text or within a figure
% environment to provide for a floating algorithm. Do NOT use the algorithm
% floating environment provided by algorithm.sty (by the same authors) or
% algorithm2e.sty (by Christophe Fiorio) as the IEEE does not use dedicated
% algorithm float types and packages that provide these will not provide
% correct IEEE style captions. The latest version and documentation of
% algorithmic.sty can be obtained at:
% http://www.ctan.org/pkg/algorithms
% Also of interest may be the (relatively newer and more customizable)
% algorithmicx.sty package by Szasz Janos:
% http://www.ctan.org/pkg/algorithmicx


% *** ALIGNMENT PACKAGES ***
%
%\usepackage{array}
% Frank Mittelbach's and David Carlisle's array.sty patches and improves
% the standard LaTeX2e array and tabular environments to provide better
% appearance and additional user controls. As the default LaTeX2e table
% generation code is lacking to the point of almost being broken with
% respect to the quality of the end results, all users are strongly
% advised to use an enhanced (at the very least that provided by array.sty)
% set of table tools. array.sty is already installed on most systems. The
% latest version and documentation can be obtained at:
% http://www.ctan.org/pkg/array

% IEEEtran contains the IEEEeqnarray family of commands that can be used to
% generate multiline equations as well as matrices, tables, etc., of high
% quality.

% *** SUBFIGURE PACKAGES ***
%\ifCLASSOPTIONcompsoc
%  \usepackage[caption=false,font=footnotesize,labelfont=sf,textfont=sf]{subfig}
%\else
%  \usepackage[caption=false,font=footnotesize]{subfig}
%\fi
% subfig.sty, written by Steven Douglas Cochran, is the modern replacement
% for subfigure.sty, the latter of which is no longer maintained and is
% incompatible with some LaTeX packages including fixltx2e. However,
% subfig.sty requires and automatically loads Axel Sommerfeldt's caption.sty
% which will override IEEEtran.cls' handling of captions and this will result
% in non-IEEE style figure/table captions. To prevent this problem, be sure
% and invoke subfig.sty's "caption=false" package option (available since
% subfig.sty version 1.3, 2005/06/28) as this is will preserve IEEEtran.cls
% handling of captions.
% Note that the Computer Society format requires a sans serif font rather
% than the serif font used in traditional IEEE formatting and thus the need
% to invoke different subfig.sty package options depending on whether
% compsoc mode has been enabled.
%
% The latest version and documentation of subfig.sty can be obtained at:
% http://www.ctan.org/pkg/subfig

% *** FLOAT PACKAGES ***
%
%\usepackage{fixltx2e}
% fixltx2e, the successor to the earlier fix2col.sty, was written by
% Frank Mittelbach and David Carlisle. This package corrects a few problems
% in the LaTeX2e kernel, the most notable of which is that in current
% LaTeX2e releases, the ordering of single and double column floats is not
% guaranteed to be preserved. Thus, an unpatched LaTeX2e can allow a
% single column figure to be placed prior to an earlier double column
% figure.
% Be aware that LaTeX2e kernels dated 2015 and later have fixltx2e.sty's
% corrections already built into the system in which case a warning will
% be issued if an attempt is made to load fixltx2e.sty as it is no longer
% needed.
% The latest version and documentation can be found at:
% http://www.ctan.org/pkg/fixltx2e

%\usepackage{stfloats}
% stfloats.sty was written by Sigitas Tolusis. This package gives LaTeX2e
% the ability to do double column floats at the bottom of the page as well
% as the top. (e.g., "\begin{figure*}[!b]" is not normally possible in
% LaTeX2e). It also provides a command:
%\fnbelowfloat
% to enable the placement of footnotes below bottom floats (the standard
% LaTeX2e kernel puts them above bottom floats). This is an invasive package
% which rewrites many portions of the LaTeX2e float routines. It may not work
% with other packages that modify the LaTeX2e float routines. The latest
% version and documentation can be obtained at:
% http://www.ctan.org/pkg/stfloats
% Do not use the stfloats baselinefloat ability as the IEEE does not allow
% \baselineskip to stretch. Authors submitting work to the IEEE should note
% that the IEEE rarely uses double column equations and that authors should try
% to avoid such use. Do not be tempted to use the cuted.sty or midfloat.sty
% packages (also by Sigitas Tolusis) as the IEEE does not format its papers in
% such ways.
% Do not attempt to use stfloats with fixltx2e as they are incompatible.
% Instead, use Morten Hogholm'a dblfloatfix which combines the features
% of both fixltx2e and stfloats:
%
% \usepackage{dblfloatfix}
% The latest version can be found at:
% http://www.ctan.org/pkg/dblfloatfix

% *** PDF, URL AND HYPERLINK PACKAGES ***
%
%\usepackage{url}
% url.sty was written by Donald Arseneau. It provides better support for
% handling and breaking URLs. url.sty is already installed on most LaTeX
% systems. The latest version and documentation can be obtained at:
% http://www.ctan.org/pkg/url
% Basically, \url{my_url_here}.

% *** Do not adjust lengths that control margins, column widths, etc. ***
% *** Do not use packages that alter fonts (such as pslatex).         ***
% There should be no need to do such things with IEEEtran.cls V1.6 and later.
% (Unless specifically asked to do so by the journal or conference you plan
% to submit to, of course. )

% correct bad hyphenation here
\hyphenation{op-tical net-works semi-conduc-tor}

\usepackage{hyperref}

\begin{document}
%
% paper title
% Titles are generally capitalized except for words such as a, an, and, as,
% at, but, by, for, in, nor, of, on, or, the, to and up, which are usually
% not capitalized unless they are the first or last word of the title.
% Linebreaks \\ can be used within to get better formatting as desired.
% Do not put math or special symbols in the title.
\title{Encryption Algorithms for Secure Communication on the Internet}

 
% author names and affiliations
% use a multiple column layout for up to three different
% affiliations
\author{\IEEEauthorblockN{Prof. Dr. Thomas ENGEL\IEEEauthorrefmark{1},
PhD Student Asya MITSEVA\IEEEauthorrefmark{2}, BICS student: Desislava MARINOVA \IEEEauthorrefmark{3}}
\IEEEauthorblockA{Faculty of Science, Technology and Communication,
University of Luxembourg,\\
Luxembourg\\
\\
Email: \IEEEauthorrefmark{1}thomas.engel@uni.lu,
\IEEEauthorrefmark{2}asya.mitseva@uni.lu,
\IEEEauthorrefmark{3}desislava.marinova.002@student.uni.lu}}


% conference papers do not typically use \thanks and this command
% is locked out in conference mode. If really needed, such as for
% the acknowledgment of grants, issue a \IEEEoverridecommandlockouts
% after \documentclass

% for over three affiliations, or if they all won't fit within the width
% of the page (and note that there is less available width in this regard for
% compsoc conferences compared to traditional conferences), use this
% alternative format:
% 
%\author{\IEEEauthorblockN{Michael Shell\IEEEauthorrefmark{1},
%Homer Simpson\IEEEauthorrefmark{2},
%James Kirk\IEEEauthorrefmark{3}, 
%Montgomery Scott\IEEEauthorrefmark{3} and
%Eldon Tyrell\IEEEauthorrefmark{4}}
%\IEEEauthorblockA{\IEEEauthorrefmark{1}School of Electrical and Computer Engineering\\
%Georgia Institute of Technology,
%Atlanta, Georgia 30332--0250\\ Email: see http://www.michaelshell.org/contact.html}
%\IEEEauthorblockA{\IEEEauthorrefmark{2}Twentieth Century Fox, Springfield, USA\\
%Email: homer@thesimpsons.com}
%\IEEEauthorblockA{\IEEEauthorrefmark{3}Starfleet Academy, San Francisco, California 96678-2391\\
%Telephone: (800) 555--1212, Fax: (888) 555--1212}
%\IEEEauthorblockA{\IEEEauthorrefmark{4}Tyrell Inc., 123 Replicant Street, Los Angeles, California 90210--4321}}




% use for special paper notices
%\IEEEspecialpapernotice{(Invited Paper)}




% make the title area
\maketitle

% As a general rule, do not put math, special symbols or citations
% in the abstract
\begin{abstract}
This document is a template for the scientific and technical report that is to be delivered by any BiCS student at the end of each Bachelor Semester Project (BSP). The Latex source files are available at:\\
\href{https://github.com/nicolasguelfi/lu.uni.course.bics.global}{{\underline{\textbf{https://github.com/nicolasguelfi/lu.uni.course.bics.global}}}}\\
  
This template is to be used using the Latex document preparation system or using any document preparation system. The whole document should be in between 6000 to 8000 words and the proportions must be preserved. The other documents to be delivered (summaries, \ldots) should have their format adapted from this template.


\end{abstract}

% no keywords

% For peer review papers, you can put extra information on the cover
% page as needed:
% \ifCLASSOPTIONpeerreview
% \begin{center} \bfseries EDICS Category: 3-BBND \end{center}
% \fi
%
% For peerreview papers, this IEEEtran command inserts a page break and
% creates the second title. It will be ignored for other modes.
\IEEEpeerreviewmaketitle

\section{Introduction ($\pm$ 5\% total words)}
% no \IEEEPARstart
We are living in the information age enabled by computing and communication technologies whose rapid evolution is almost taken for granted today. This is an age dense with electronic connectivity, eavesdropping and fraud. We send messages around the world instantaneously and we do not want them to be intercepted or stolen. Thus, information is an asset that has precious value like any other. Society’s reliance on a changing panoply of information technologies and technology-enabled services, the increasing global nature of commerce and business, and the ongoing desire to protect freedoms all suggest that future needs for data security is vital.  Network – based attacks spring up every day, distance does not matter anymore since data has been digitalized and there is the issue of privacy being collected, processed and misrepresented. For this reason, the disciplines of cryptography and network security have matured. Thus, internet security is complex yet fascinating. To develop a security mechanism (algorithm or protocol) one must always consider potential attacks, take countermeasures, act like an intruder and go against one’s own intuition and exploit eventual weaknesses in the algorithm. 

We dive into the terminology that governs our topic in the next paragraph. The word “cryptography” comes from two Greek words meaning “secret writing” [1]. Cryptography is a vast, rich subject. It is the art and science of concealing meaning, of keeping the enciphered information secret and of designing and analyzing encryption schemes. It involves mathematics as a supporting tool. Cryptanalysis is the breaking of codes and uses statistical and mathematical approaches. The areas of cryptography and cryptanalysis together are called cryptology. The basic component of cryptography is a cryptosystem which is based on substitution (it changes characters in the plaintext to produce the ciphertext) and permutation (an ordered sequence of elements in a finite set with each character appearing only once).
If someone wants to break the ciphertext, knowing the algorithm but not the specific cryptographic key,  there are 3 main types of attacks [Ref: Stallings, 7th ed, p.90]: 
-	ciphertext only attack – ciphertext known, the goal is to find the corresponding plaintext. The key may be found as well if possible. (A Caesar cipher is susceptible to a statistical ciphertext-only attack)
-	known plaintext attack – both ciphertext and plaintext and known; the goal is to find the key
-	chosen plaintext attack – specific plaintexts are enciphered in order to find out the corresponding ciphertexts and the goal is to find the key.
	a good cryptosystem protects against all three types of attacks. 

We often stumble upon impending security threats such as information leakage; integrity violation; denial of service; illegitimate usage; trojan horses; insider attacks; problems related to access or authentication control. 
Thus, there are instilled security requirements of utmost importance. We present them as follows: confidentiality; integrity; authentication; availability; access control; non-repudiation, which are often intermingled. Hence, one of the mechanisms promising to ensure security is encryption, along with digital signatures and hashing. 

In order to make information secret, we use a cipher – an algorithm that converts plain text into ciphertext, which is in unreadable form, unless we have a key that lets us convert back the cipher. The process of making text secret is called encryption, and the reverse process is called decryption.

There are two types of encryption: symmetric and asymmetric. For this project, we shall focus mainly on the symmetric, also called “conventional”. It comprises a single key for both encryption and decryption and it is the preferred choice when we need to encrypt large amounts of data. On the other hand, asymmetric cryptography, also called “public key” and “two-key”, consists of two keys: public and private. The public key is used to encrypt the message, whereas the private key is used to reverse the encryption. This method cannot deal with large quantities of data due to the runtime of supporting processes employed. However, it is more secure when we want to transfer over an insecure channel. Nevertheless, the choice to be made on which system to use for encryption and decryption depends on the purposes and on each algorithms’ functions. 


\section{Project description ($\pm$ 10\% total words) }
\subsection{Domain}
The report draws on a variety of disciplines. However, it is impossible to appreciate the vastness and significance of the topic in a limited amount of content pages. Nevertheless, we attempt at making the report self-contained by providing the reader with an intuitive understanding of our survey and application results.
The domains that are associated with our project are: 
-	Internet security: "measures to protect data during its transmission over a collection of interconnected networks" [Stallings, ed.3,]
-	Computer Security: "generic name for the collection of tools designed to protect data and to thwart hackers and malicious software (viruses) " [Stallings, ed.3,]; at its heart there are three key objectives: confidentiality, integrity and availability
-	Cryptographic primitives
-	Symmetric-key cryptography
-	cryptographic algorithms: "study of techniques for ensuring the secrecy and/or authenticity of information => 3 main studies of this category (1) symmetric encryption, (2) asymmetric encryption and (3) cryptographic hash functions" [Stallings, 5th ed, p. 3]



\subsection{Objectives}
The objectives of this project are three-fold. It We aim for a comprehensive survey of both principles and practice of cryptography and network security. In a first part, we address a highly needed background material.  Our main focus falls on symmetric encryption, including classical and modern algorithms. We give a basic introduction to the concepts of block ciphers and stream ciphers and how they obtain their security. We put emphasis on DES, AES (+ other??) algorithms. Then, we implement an encrypted chat system that could be of use to provide network security.
\subsection{Constraints}
Our project takes into account substitution/transposition; block/stream ciphers but we do not consider:
-	Asymmetric ciphers (public-key algorithms, including RSA and elliptic curve); 
-	Data integrity algorithms (cryptographic hash functions (message authentication codes (MACs); digital signatures)); 
-	Authentication techniques: key management and key distribution topics (protocols for key exchange)



\section{BSPro - A First Bachelor Semester Project in BiCS-land}
\subsection{Requirements ($\pm$ 15\% total words)}
[[[[Describe here all the properties that characterize the deliverables you produce. It should describe what are those deliverables, who are the actors exploiting the deliverables, what are the expected functional and non functional qualities of the deliverables.]]]]

Ciphers have been used long before apparition of computers. During Roman times, Julius Caesar invented an encryption for his private correspondence. Today, it is known as the Caesar cipher. Given the English alphabet, the cipher shifts a letter from the alphabet three places to the right, i.e. A is encrypted as D, B as E,etc. It is a monoalphabetic cipher. To decrypt the message, the other party needs to know both the algorithm and the shifting number. The secret key shared by the sender and the recipient of the message is k=3. Breaking the Caesar Cipher can be done by testing all possible shifts. Since an alphabet of length 26 is used we have to test 26 shifts. 

ROT13 is a simple monoalphabetic substitution cipher, a special class of Caesar cipher, that encodes a certain letter with another letter that is 13 positions after it. Only those letters which occur in the English alphabet are affected. Numbers, symbols, whitespace, and all other characters are left unchanged. It is an example of a cipher providing weak encryption, since both operations encryption and decryption are identical. Hence, this cipher is its own inverse. Because we know there are 26 letters in the English alphabet, if we wish to apply twice 13 it would give us one shift of 26. Thus, it leads us back to the original text. Moreover, the direction of the shift is of no importance, since it will always give the same output \cite{swenson2008modern}.
ROT13 is used in online forums as a means of hiding spoilers, punchlines, puzzle solutions, and offensive materials from the casual glance. Furthermore, it has inspired a variety of letter and word games online \cite{wikirot13}. 

Another cipher our project deals with is the block cipher Data Encryption Standard (DES). In DES we put 64-bit block of plaintext,  DES' key (which does the processing) is 64 bit which is 8 bytes but for each byte there is one parity bit, therefore, the value in the key is only 56 bits which means there are 2 to the power of 56 different keys. The output ciphertext is a 64-bit block. 
From the 56-bit key, 16 bits are generated (one for each round). Each DES round works consecutively, has the same operations and uses a different key. Each round uses a combination of proper substitution, where we take some bits which are substituted with another combination of bits. Each DES round takes as an input the ciphertext produced by the previous round and outputs the ciphertext for the next round. The input is divided into a left half and a right half. The output left half is just the right half of the input. The right output is the result of XOR-ing the left half of the input and the output of the Mangler Function (which takes as an input the 32-bit right half, expands it to 48-bit (bit-wise operation) then XORs it to 48-bit key, then uses the S-Boxes to substitute the 48-bit value into a 32-bit value).
The algorithm process of decryption in DES is the same as the encryption process. It uses the ciphertext as an input to DES but the keys are run in reversed order, i.e. k = 16 is used as the first round of decryption, k = 15 is used as a second round of decryption and so on, so forth. 
Diffusion is one of the principles in encryption. It is achieved through permutation (initial transposition). Permutation works by changing the position of the bits in DES. The mixed bits are taken to a sub-box which receives the 56-bit key and the 64-bit plaintext, once it completes processing, it outputs the 64 bits into another transposition subsection, which in turn produces a 64-bit ciphertext. 
The larger the block size, the key size and number of rounds means greater security. However, when there are more than 16 rounds or more than 56 keys, the security will neither be increased nor the encryption will be made any stronger. The possible attacks that often occur on product ciphers are differential cryptanalysis and linear cryptanalysis, however, according to Stallings, DES has proven to be resistant to these sort of attacks \cite{stallings2011-a}. By 1999, a computer could try every possible key in a couple of days rendering the cipher insecure.

\subsection{Design ($\pm$ 20\% total words)}

The technical part of our project consists of creating a chat application. Our source code editor is Visual Studio Code and we program in Python. We create a TCP server as well as a TCP client that connects to the server. 
For the client to use TCP Networking, we require a socket module that serves as a communication endpoint with two parameters:
s= socket.socket(socket.AF_INET, socket.SOCK_STREAM)
the first parameter AF_INET stands for IPv4, the second parameter \textit{SOCK_STREAM} means we are going to use TCP. We specify a variable for port number (on which the server is running) and ip, which holds the ip of the host machine it is running on. We use \textit{gethostbyname} to look for the host’s ip since the host machine does not have a fixed IP address and often when it reconnects to an Internet connection is assigned a new IP address.
\textit{ip = str(socket.gethostbyname(socket.gethostname()))}
\textit{port = 1234}
Once we the port and ip address are known, we can connect to the server by passing the function to a data structure \textit{s.connect(((ip, port))}

In the Python script for the server, we turn the connections into threads creating a multithreaded server. A thread is an external process and we can have many running simultaneously. This facilitates the whole process when there are many clients that connect to the server. We import the thread library import threading. We create an EchoThread class and pass to its function a connection and address. For the server to use TCP Networking, we require a socket module that serves as a communication endpoint with two parameters:
\textit{s= socket.socket(socket.AF_INET, socket.SOCK_STREAM)}
There is also a callback to every connection established to that server with variable “conn”, short for connection. It describes information about the connection. TCP allows us to open a connection, execute some data transfer on it without even closing the connection which may stay open for a long period of time. We subscribe to two events. The first one is data and second one is closing the connection. For the data event, each time data is sent over to the server, data is received by the client. Once the client cuts the connection or the EchoThread is terminated, the connection is ended.
For the client to work, a port number is specified equal to the server’s so that both can connect. When the port and host have been bounded, we can listen to incoming connections with socket variable s and calling listen on it: \textit{s.listen(10)}. We pass the variable 10 which means that up to 10 people can be queued for us to handle the requests. However, the eleventh will be rejected. To accept requests from outside, we use connection variable and address variable as: \textit{conn, addr = s.accept()} which stores the ip import of the client trying to connect and we call accept on the socket we have created. Upon a connection being established, the client can send messages to the server that will cause it to echo back the message. The server can receive data from client by creating variable data and sending it to the response gathered from connection. We write: \textit{data = conn.recv(4096)}. Then server sends that message to all connected clients \textit{conn.sendall(data}. Afterwards we can close the socket connection between client and server \textit{print("connection closed with ", addr)} as well as close the socket itself \textit{s.close()} that allows connections to exist.
For our project, we aim at implementing in Python several encryption and decryption applications. We make use of Caeser Cipher, ROT13, DES and AES.
DES and AES have been implemented through the means of a crypto library “PyCryptodome”. It is a self-contained Python package of low-level cryptographic primitives. It supports PyPy, Python2 and Python3. PyCryptodome can be used as a replacement for the old PyCrypto library. We install all modules under Crypto package with pip3 install pycryptodome.  However, having both PyCrypto and PyCryptodome installed at the same time is not a good idea as they will interfere with each other. If, however, one insists on having them both, it would be best to deploy them in a virtual environment. To have an independent library of the old PyCrypto, it suffices to type the following command in Terminal shell:\textit{pip3 install pycryptodomex} and all modules are installed under Cryptodome package and PyCrypto and PyCryptodome can coexist. 


\subsection{Production ($\pm$ 20\% total words)}

We implement all algorithms with Pythong Programming language. 

Caeser Cipher is a substitution cipher and a type of shift cipher. Shift ciphers work by using the modulo operator to encrypt and decrypt messages. In the following lines we demonstrate its algorithm. We take an alphabetic message (A to Z). We take a key, an integer from 0 to 25, equal to 3.To encrypt, either left-shift or right-shift letter by letter by the value of key. For instance, if message is ABC and key K is 3, we perform right-shift and the encrypted text is A + 3 = C, B+3 = E, C+3 = F, hence, CEF. To decrypt, left-shift or right-shift the message letter by letter opposite to the shift performed in encryption by the value of key. For example if encrypted text is "CEF" and key is 3 and we are performing left-shift then encrypted text will be C-3 =A , E-3 =B B, F-3 = C, hence, ABC being decrypted text. 
For every letter (char) in the message, we find the letter that matches its order in the alphabet starting from 0. We calculate \textit{newPosition = (position + key) mod 26}, then convert \textit{newPosition} into a letter that matches its order in the alphabet starting from 0. The decryption process is similar; however, we subtract the key during the modular operation.

To implement DES, we import DES function from Cryptodome library module. Then we create and specify a \textit{key = “mysecret”}, a password that shall be used to encrypt the text and is 8 bytes long. We have created a pad function taking as input text. Whatever we type for the text parameter, the pad function shall operate on the text. We run a while loop checking the length of text mod 8, i.e. the modulus divides it and shows the remainder. If remainder is not equal to zero, we add a space at the end of text such that it becomes multiple of 8 (e.g. 8, 16, 24, 32, etc.), otherwise the encryption is not going to work as DES encryption algorithm takes input in 8 bytes. For instance, if we type ‘abcd’ it is not going to work, hence, the function adds 4 empty spaces at the end of ‘abcd’ to make sure it is a total of 8 characters/bytes. Then, we create an object, called ‘des’. We use \textit{new()} function of module DES that takes two parameters: key and \textit{DES.MODE_ECB} which is the encryption mode. We have created a variable taking an input from client that shall be encrypted. The \textit{padded_text} variable uses the pad() function on the input to make sure the input provided by client is a multiple of 8 and if necessary adds empty spaces to conform to the multiple condition. Next, we have created an \textit{encrypted_text} variable that uses the \textit{encrypt()} function on the \textit{padded_text} and turns it into an encrypted string. In order to decrypt, we run des submodule and pass decrypt() function on it. 



\subsection{Assessment ($\pm$ 15\% total words)}
Provide any objective elements to assess that your deliverables reached or not the requirements described above. 
\section*{Acknowledgment}
The authors would like to thank the BiCS management and education team for the amazing work done.

\section{Conclusion}
The conclusion goes here.


\newpage 
\section{Appendix}
All images and additional material go there.

A link to the GitHub repository of the project: \url{https://github.com/dmarinova1/BSP-S3-Encryption-algorithms}
% that's all folks


% An example of a floating figure using the graphicx package.
% Note that \label must occur AFTER (or within) \caption.
% For figures, \caption should occur after the \includegraphics.
% Note that IEEEtran v1.7 and later has special internal code that
% is designed to preserve the operation of \label within \caption
% even when the captionsoff option is in effect. However, because
% of issues like this, it may be the safest practice to put all your
% \label just after \caption rather than within \caption{}.
%
% Reminder: the "draftcls" or "draftclsnofoot", not "draft", class
% option should be used if it is desired that the figures are to be
% displayed while in draft mode.
%
%\begin{figure}[!t]
%\centering
%\includegraphics[width=2.5in]{myfigure}
% where an .eps filename suffix will be assumed under latex, 
% and a .pdf suffix will be assumed for pdflatex; or what has been declared
% via \DeclareGraphicsExtensions.
%\caption{Simulation results for the network.}
%\label{fig_sim}
%\end{figure}

% Note that the IEEE typically puts floats only at the top, even when this
% results in a large percentage of a column being occupied by floats.


% An example of a double column floating figure using two subfigures.
% (The subfig.sty package must be loaded for this to work.)
% The subfigure \label commands are set within each subfloat command,
% and the \label for the overall figure must come after \caption.
% \hfil is used as a separator to get equal spacing.
% Watch out that the combined width of all the subfigures on a 
% line do not exceed the text width or a line break will occur.
%
%\begin{figure*}[!t]
%\centering
%\subfloat[Case I]{\includegraphics[width=2.5in]{box}%
%\label{fig_first_case}}
%\hfil
%\subfloat[Case II]{\includegraphics[width=2.5in]{box}%
%\label{fig_second_case}}
%\caption{Simulation results for the network.}
%\label{fig_sim}
%\end{figure*}
%
% Note that often IEEE papers with subfigures do not employ subfigure
% captions (using the optional argument to \subfloat[]), but instead will
% reference/describe all of them (a), (b), etc., within the main caption.
% Be aware that for subfig.sty to generate the (a), (b), etc., subfigure
% labels, the optional argument to \subfloat must be present. If a
% subcaption is not desired, just leave its contents blank,
% e.g., \subfloat[].


% An example of a floating table. Note that, for IEEE style tables, the
% \caption command should come BEFORE the table and, given that table
% captions serve much like titles, are usually capitalized except for words
% such as a, an, and, as, at, but, by, for, in, nor, of, on, or, the, to
% and up, which are usually not capitalized unless they are the first or
% last word of the caption. Table text will default to \footnotesize as
% the IEEE normally uses this smaller font for tables.
% The \label must come after \caption as always.
%
%\begin{table}[!t]
%% increase table row spacing, adjust to taste
%\renewcommand{\arraystretch}{1.3}
% if using array.sty, it might be a good idea to tweak the value of
% \extrarowheight as needed to properly center the text within the cells
%\caption{An Example of a Table}
%\label{table_example}
%\centering
%% Some packages, such as MDW tools, offer better commands for making tables
%% than the plain LaTeX2e tabular which is used here.
%\begin{tabular}{|c||c|}
%\hline
%One & Two\\
%\hline
%Three & Four\\
%\hline
%\end{tabular}
%\end{table}


% Note that the IEEE does not put floats in the very first column
% - or typically anywhere on the first page for that matter. Also,
% in-text middle ("here") positioning is typically not used, but it
% is allowed and encouraged for Computer Society conferences (but
% not Computer Society journals). Most IEEE journals/conferences use
% top floats exclusively. 
% Note that, LaTeX2e, unlike IEEE journals/conferences, places
% footnotes above bottom floats. This can be corrected via the
% \fnbelowfloat command of the stfloats package.

% trigger a \newpage just before the given reference
% number - used to balance the columns on the last page
% adjust value as needed - may need to be readjusted if
% the document is modified later
%\IEEEtriggeratref{8}
% The "triggered" command can be changed if desired:
%\IEEEtriggercmd{\enlargethispage{-5in}}

% references section

% can use a bibliography generated by BibTeX as a .bbl file
% BibTeX documentation can be easily obtained at:
% http://mirror.ctan.org/biblio/bibtex/contrib/doc/
% The IEEEtran BibTeX style support page is at:
% http://www.michaelshell.org/tex/ieeetran/bibtex/
%\bibliographystyle{IEEEtran}
% argument is your BibTeX string definitions and bibliography database(s)
%\bibliography{IEEEabrv,../bib/paper}
%
% <OR> manually copy in the resultant .bbl file
% set second argument of \begin to the number of references
% (used to reserve space for the reference number labels box)

\end{document}
