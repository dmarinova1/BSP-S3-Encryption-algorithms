\section{Project description ($\pm$ 10\% total words) }
\subsection{Domain}
The report draws on a variety of disciplines. However, it is impossible to appreciate the vastness and significance of the topic in a limited amount of content pages. Nevertheless, we attempt to make the report self-contained by providing the reader with an intuitive understanding of our survey and application results.
The domains that are associated with our project are: 
\begin{itemize}
\item Internet security: a branch to Computer Security, related to the Internet,comprised of measures to protect transactions done over a collection of interconnected networks
\end {itemize}
\begin{itemize} 
\item Computer Security: the process of preventing and detecting unauthorized use of our computer 
\end{itemize}
\begin{itemize}
\item Cryptographic primitives:  well-established, low-level cryptographic algorithms frequently used to build cryptographic protocols for computer security systems
\end{itemize}
\begin{itemize}
\item Symmetric-key cryptography: same key is used for both encryption and decryption
\end{itemize}
\begin{itemize}
\item Cryptographic algorithms: ensure alteration of data in order to preserve its authenticity, confidentiality, integrity through various security mechanisms such as symmetric encryption,  asymmetric encryption and cryptographic hash functions
\end{itemize}




\subsection{Objectives}
We aim for a comprehensive survey of both principles and practice of cryptography and network security. The objectives of this project are three-fold. 
\begin{enumerate}
\item  As a preliminary task, we address a highly needed background material. We give a basic introduction to the concepts and primitives of cryptography, i.e. transposition/permutation, encryption/decryption, symmetry/asymmetry, block ciphers/stream ciphers and how they obtain their security in constantly emerging attacks
\item  Our main focus falls on symmetric encryption, taking into account both classical and modern algorithms. Therefore,  we explore the following encryption algorithms: Caesar cipher, ROT13, One-time pad, DES, 3DES, AES in order to yield two types of approaches: information theoretic and computational
\item Lastly, we implement an encrypted chat system that could be of use to provide network security.
\end{enumerate}


\subsection{Constraints}
Our project takes into account substitution/transposition; block/stream ciphers but we do not consider:
\begin{itemize}
\item Asymmetric ciphers (public-key algorithms, including RSA and elliptic curve); 
\end{itemize}
\begin{itemize}
\item Data integrity algorithms (cryptographic hash functions, message authentication codes (MACs); digital signatures)); 
\end{itemize}
\begin{itemize}
\item Authentication techniques: key management and key distribution topics (protocols for key exchange);
\end{itemize}


